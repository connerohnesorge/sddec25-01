\chapter{Introduction}\label{chap:introduction}

\section{Problem Statement}\label{sec:problem-statement}

Individuals with mobility impairments and underlying conditions face
the challenge of detecting and responding to medical episodes before
they occur, which can happen anytime and anywhere, posing risks to
their safety and independence. Current assistive
technologies often fail to proactively ensure user well-being.
Current healthcare solutions are reactive, requiring human
intervention after an episode occurs, which can lead to delayed
response times, severe medical complications, and loss of autonomy.

Advances in artificial intelligence and edge computing offer new opportunities
for real-time health monitoring~\cite{lakshminarayanan2023health}, yet these
technologies remain underutilized in assistive mobility devices. The ability to predict and respond to medical emergencies in
real-time would not only enhance personal safety but also reduce the
burden on caregivers and emergency medical services, improving
overall healthcare efficiency.

Our project leverages semantic segmentation at the edge to analyze
physiological indicators such as eye movement and body posture. Neurological literature establishes these indicators: abnormal eyelid
dynamics contain seizure-specific anomalies~\cite{sedighsarvestani2012eyelid},
and blink reflex patterns link to neurological state through motor control
pathways~\cite{evinger2011blinking}. Eyelid myoclonia has been
identified as a distinct ictal sign in idiopathic generalized
epilepsy~\cite{stefan2007eyelid}, demonstrating that eye-related
movements provide clinically meaningful indicators of seizure
activity. Prior work demonstrates that eye and head motion patterns serve as
seizure indicators~\cite{provost2022eye}. Integrating this technology
into wheelchairs creates an intelligent system that detects early warning
signs and autonomously repositions the user before critical incidents occur,
using established U-Net architectures for biomedical segmentation~\cite{ronneberger2015}.

\section{Clarification of Original Project Scope}\label{sec:original-scope}

The following is the exact project description provided by the client,
JR Spidell, to Iowa State University in the approved senior design
project list. This description was submitted by the client and has not
been edited or construed by our team:

\begin{quote}
  \textit{``This project will break up an existing U-Net model into code
    segments that can then be pipelined. The result will be slightly
    higher latency but also higher throughput of the algorithm.''}
\end{quote}

\textbf{Technical Clarification:} The original description suggests that
pipelining would enable multiple simultaneous inferences through the
Deep Processing Unit (DPU), which is technically inaccurate for the AMD
Kria KV260 platform. The DPU architecture fundamentally constrains
execution to one inference at a time---it cannot process multiple neural
network inferences simultaneously.

Our actual implementation leverages pipelined CPU processing stages
(preprocessing and postprocessing) that operate concurrently with
sequential DPU execution. This approach achieves higher throughput
through overlapped CPU and DPU operations rather than parallel DPU
inferences. The architecture maintains a single-threaded DPU execution
path with multi-threaded CPU stages, enabling 60 FPS performance while
respecting the hardware's sequential processing constraint.

This clarification is necessary to establish accurate expectations and
demonstrate our team's understanding of the platform's technical
limitations. The confusion in the original description does not
reflect the team's knowledge or capabilities but rather stems from the
initial project proposal submitted to the university.

\section{Intended Users}\label{sec:intended-users}

The system serves three user groups: (1) wheelchair users with conditions
such as cerebral palsy, epilepsy, or cardiovascular disorders requiring
autonomous episode detection; (2) caregivers needing real-time health alerts
without constant supervision; and (3) healthcare providers relying on accurate
physiological data for rapid decision-making~\cite{beauchamp2007ethics}.
